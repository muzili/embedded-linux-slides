  \SlideTitle{proc virtual filesystem}
  \startitemize
  \item The \type{proc} virtual filesystem exists since the beginning of
    Linux
  \item It allows
    \startitemize
    \item The kernel to expose statistics about running processes in
      the system
    \item The user to adjust at runtime various system parameters
      about process management, memory management, etc.
    \stopitemize
  \item The \type{proc} filesystem is used by many standard userspace
    applications, and they expect it to be mounted in \type{/proc}
  \item Applications such as \type{ps} or \type{top} would not work
    without the \type{proc} filesystem
  \item Command to mount \type{/proc}:\\
    \type{mount -t proc nodev /proc}
  \item \type{filesystems/proc.txt} in the kernel sources
  \item \type{man proc}
  \stopitemize

  \SlideTitle{proc contents}
  \startitemize
  \item One directory for each running process in the system
    \startitemize
    \item \type{/proc/<pid>}
    \item \type{cat /proc/3840/cmdline}
    \item It contains details about the files opened by the process,
      the CPU and memory usage, etc.
    \stopitemize
  \item \type{/proc/interrupts}, \type{/proc/devices},
    \type{/proc/iomem}, \type{/proc/ioports} contain general
    device-related information
  \item \type{/proc/cmdline} contains the kernel command line
  \item \type{/proc/sys} contains many files that can be written to to
    adjust kernel parameters
    \startitemize
    \item They are called {\em sysctl}. See \type{sysctl/}
      in kernel sources.
    \item Example\\
      \type{echo 3 > /proc/sys/vm/drop_caches}
    \stopitemize
  \stopitemize

  \SlideTitle{sysfs filesystem}
  \startitemize
  \item The \type{sysfs} filesystem is a feature integrated in the 2.6
    Linux kernel
  \item It allows to represent in userspace the vision that the kernel
    has of the buses, devices and drivers in the system
  \item It is useful for various userspace applications that need to
    list and query the available hardware, for example \type{udev} or
    \type{mdev}.
  \item All applications using sysfs expect it to be mounted in the
    \type{/sys} directory
  \item Command to mount \type{/sys}:\\
    \type{mount -t sysfs nodev /sys}
  \item
\starttyping
$ ls /sys/
block bus class dev devices firmware
fs kernel modulepower
\stoptyping
  \stopitemize
