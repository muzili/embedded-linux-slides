  \SlideTitle{Embedded Linux solutions}
  \startitemize
  \item Two ways to switch to embedded Linux
    \startitemize
    \item Use {\bf solutions provided and supported by vendors} like
      MontaVista, Wind River or TimeSys. These solutions come with
      their own development tools and environment. They use a mix of
      open-source components and proprietary tools.
    \item Use {\bf community solutions}. They are completely open,
      supported by the community.
    \stopitemize
  \item In Free Electrons training sessions, we do not promote a particular
    vendor, and therefore use community solutions
    \startitemize
    \item However, knowing the concepts, switching to vendor solutions will be easy
    \stopitemize
  \stopitemize

  \SlideTitle{OS for Linux development}
  \startitemize
  \item We strongly recommend to use Linux as the desktop operating
    system to embedded Linux developers, for multiple reasons.
  \item All community tools are developed and designed to run on
    Linux. Trying to use them on other operating systems (Windows, Mac
    OS X) will lead to trouble, and their usage on those systems is
    generally not supported by community developers.
  \item As Linux also runs on the embedded device, all the knowledge
    gained from using Linux on the desktop will apply similarly to the
    embedded device.
  \stopitemize

  \SlideTitle{Desktop Linux distribution}
  \startitemize
    \item {\bf Any good and sufficiently recent Linux desktop
        distribution} can be used for the development workstation
      \startitemize
      \item Ubuntu, Debian, Fedora, openSUSE, Red Hat, etc.
      \stopitemize
    \item We have chosen Ubuntu, as it is a {\bf widely used and easy to
        use} desktop Linux distribution
    \item The Ubuntu setup on the training laptops has intentionally
      been left untouched after the normal installation
      process. Learning embedded Linux is also about learning the tools
      needed on the development workstation!
    \stopitemize


  \SlideTitle{Linux root and non-root users}
  \startitemize
  \item Linux is a multi-user operating system
    \startitemize
    \item The {\bf root user is the administrator}, and it can do
      privileged operations such as: mounting filesystems, configuring
      the network, creating device files, changing the system
      configuration, installing or removing software
    \item All {\bf other users are unprivileged}, and cannot perform
      those administrator-level operations
    \stopitemize
  \item On an Ubuntu system, it is not possible to log in as
    \type{root}, only as a normal user.
  \item The system has been configured so that the user account
    created first is allowed to run privileged operations through a
    program called \type{sudo}.\\
    \startitemize
    \item Example: \type{sudo mount /dev/sda2 /mnt/disk}
    \stopitemize
  \stopitemize

  \SlideTitle{Software packages}
  \startitemize
  \item The distribution mechanism for software in GNU/Linux is
    different from the one in Windows
  \item Linux distributions provides a central and coherent way of
    installing, updating and removing applications and libraries:
    {\bf packages}
  \item Packages contains the application or library files, and
    associated meta-information, such as the version and the
    dependencies
    \startitemize
    \item \type{.deb} on Debian and Ubuntu, \type{.rpm} on Red Hat,
      Fedora, openSUSE
    \stopitemize
  \item Packages are stored in {\bf repositories}, usually on HTTP or
    FTP servers
  \item You should only use packages from official repositories for your
    distribution, unless strictly required.
\stopitemize

  \SlideTitle{Managing software packages (1)}
  Instructions for Debian based GNU/Linux systems\\
  (Debian, Ubuntu...)
  \startitemize
  \item Package repositories are specified in \type{/etc/apt/sources.list}
  \item To update package repository lists:\\
    \type{sudo apt-get update}
  \item To find the name of a package to install, the best is to use
    the search engine on \hyphenatedurl{http://packages.debian.org} or on
    \hyphenatedurl{http://packages.ubuntu.com}. You may
    also use:\\
    \type{apt-cache search <keyword>}
  \stopitemize

  \SlideTitle{Managing software packages (2)}
  \startitemize
  \item To install a given package:\\
    \type{sudo apt-get install <package>}
  \item To remove a given package:\\
    \type{sudo apt-get remove <package>}
  \item To install all available package updates:\\
    \type{sudo apt-get dist-upgrade}
  \item Get information about a package:\\
    \type{apt-cache show <package>}
  \item Graphical interfaces
    Synaptic for GNOME
    KPackageKit for KDE
  \stopitemize
  Further details on package management:\\
  \hyphenatedurl{http://www.debian.org/doc/manuals/apt-howto/}

  \SlideTitle{Host vs. target}
  \startitemize
  \item When doing embedded development, there is always a split between
    \startitemize
    \item The {\em host}, the development workstation, which is
      typically a powerful PC
    \item The {\em target}, which is the embedded system under
      development
    \stopitemize
  \item They are connected by various means: almost always a serial
    line for debugging purposes, frequently an Ethernet connection,
    sometimes a JTAG interface for low-level debugging

  \stopitemize
  \placefigure[center,none]{}{
        \externalfigure[slides/sysdev-dev-environment/host-vs-target.pdf][width=0.7\textwidth]
  }

  \SlideTitle{Serial line communication program}
  \startitemize
  \item An essential tool for embedded development is a serial line
    communication program, like HyperTerminal in Windows.
  \item There are multiple options available in Linux: Minicom,
    Kermit, Picocom, Gtkterm, Putty, etc.
  \item I recommend \type{kermit}
    \startitemize
    \item Installation with \type{sudo apt-get install ckermit}
    \item Create the kermit init script \type{~/bin/usb0term}
\starttyping
#!/usr/bin/kermit
echo connecting /dev/ttyUSB0 .....
set line /dev/ttyUSB0
set modem type none   ; There is no modem
set speed 115200       ; Or other desired speed
set carrier-watch off
set prefixing all
set parity none
set stop-bits 1
set serial 8n1
set modem none
set file type bin
set file name lit
set flow-control none
set prompt "Linux Kermit> "
connect
\stoptyping
    \item Run with \type{usb0term}
    \item Exit with \type{Control-\ c} and \type{exit}
    \stopitemize
  \item \type{serial device} is typically
    \startitemize
    \item \type{ttyUSBx} for USB to serial converters
    \item \type{ttySx} for real serial ports
    \stopitemize
  \stopitemize

  \SlideTitle{Command line tips}
  \startitemize
  \item Using the command line is mandatory for many operations needed
    for embedded Linux development
  \item It is a very powerful way of interacting with the system, with
    which you can save a lot of time.
  \item Some useful tips
    \startitemize
    \item You can use several tabs in the Gnome Terminal
    \item Remember that you can use relative paths (for example:
      \type{../../linux}) in addition to absolute paths (for example:
      \type{/home/user})
    \item In a shell, hit \type{[Control] [r]}, then a keyword, will
      search through the command history. Hit \type{[Control] [r]}
      again to search backwards in the history
    \item You can copy/paste paths directly from the file manager to
      the terminal by drag-and-drop.
    \stopitemize
  \stopitemize


