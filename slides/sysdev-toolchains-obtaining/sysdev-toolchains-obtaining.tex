
  \SlideTitle{Building a toolchain manually}

  Building a cross-compiling toolchain by yourself is a difficult and painful
  task! Can take days or weeks!
  \startitemize
  \item Lots of details to learn: many components to build, complicated
    configuration
  \item Lots of decisions to make (such as C library version, ABI, floating point
    mechanisms, component versions)
  \item Need kernel headers and C library sources
  \item Need to be familiar with current \type{gcc} issues and patches
    on your platform
  \item Useful to be familiar with building and configuring tools
  \item See the {\em Crosstool-NG} \type{docs/} directory for details
    on how toolchains are built.
\stopitemize

  \SlideTitle{Get a pre-compiled toolchain}
  \startitemize
  \item Solution that many people choose
    \startitemize
    \item Advantage: it is the simplest and most convenient solution
    \item Drawback: you can't fine tune the toolchain to your needs
    \stopitemize
  \item Determine what toolchain you need: CPU, endianism, C library, component
    versions, ABI, soft float or hard float, etc.
  \item Check whether the available toolchains match your requirements.
  \item Possible choices
    \startitemize
    \item Sourcery CodeBench toolchains
    \item Linaro toolchains
    \item More references at \hyphenatedurl{http://elinux.org/Toolchains}
    \stopitemize
  \stopitemize

  \SlideTitle{Sourcery CodeBench}
  \startitemize
  \item {\em CodeSourcery} was a a company with an extended expertise
    on free software toolchains: gcc, gdb, binutils and glibc. It has
    been bought by {\em Mentor Graphics}, which continues to provide
    similar services and products
  \item They sell toolchains with support, but they also provide a
    ”{\em Lite}” version, which is free and usable for commercial
    products
  \item They have toolchains available for
    \startitemize
    \item ARM
    \item MIPS
    \item PowerPC
    \item SuperH
    \item x86
    \stopitemize
  \item Be sure to use the GNU/Linux versions. The EABI versions are
    for bare-metal development (no operating system)
  \stopitemize

  \SlideTitle{Linaro toolchains}
    \placefigure[right,none]{}{\externalfigure[slides/sysdev-toolchains-obtaining/linaro.png][width=0.1\textwidth]}
    \startitemize
    \item Linaro contributes to improving mainline gcc on ARM, in
      particular by hiring CodeSourcery developers.
    \item For people who can't wait for the next releases of gcc, Linaro
      releases modified sources of stable releases of gcc, with these
      optimizations for ARM (mainly for recent Cortex A CPUs).
    \item As any gcc release, these sources can be used by build tools
      to build their own binary toolchains (Buildroot, OpenEmbedded...)
      This allows to support glibc, uClibc and eglibc.
    \item \hyphenatedurl{https://wiki.linaro.org/WorkingGroups/ToolChain}
    \item Binary packages are available for Ubuntu users,
      \hyphenatedurl{https://launchpad.net/~linaro-maintainers/+archive/toolchain}
    \stopitemize


  \SlideTitle{Installing and using a pre-compiled toolchain}
  \startitemize
  \item Follow the installation procedure proposed by the vendor
  \item Usually, it is simply a matter of extracting a tarball
        wherever you want.
  \item Then, add the path to toolchain binaries in your \type{PATH}:\\
    \type{export PATH=/path/to/toolchain/bin/:$PATH}
  \item Finally, compile your applications\\
    \type{PREFIX-gcc -o foobar foobar.c}
  \item \type{PREFIX} depends on the toolchain configuration, and
    allows to distinguish cross-compilation tools from native
    compilation utilities
  \stopitemize

  \SlideTitle{Toolchain building utilities}
  Another solution is to use utilities that {\bf automate the process of
  building the toolchain}
  \startitemize
  \item Same advantage as the pre-compiled toolchains: you don't need
    to mess up with all the details of the build process
  \item But also offers more flexibility in terms of toolchain
    configuration, component version selection, etc.
  \item They also usually contain several patches that fix known
    issues with the different components on some architectures
  \item Multiple tools with identical principle: shell scripts or
    Makefile that automatically fetch, extract, configure, compile and
    install the different components
\stopitemize

  \SlideTitle{Toolchain building utilities (2)}
  \startitemize
  \item {\bf Crosstool-ng}
    \startitemize
    \item Rewrite of the older Crosstool, with a menuconfig-like configuration
      system
    \item Feature-full: supports uClibc, glibc, eglibc, hard and soft
      float, many architectures
    \item Actively maintained
    \item \hyphenatedurl{http://crosstool-ng.org/}
    \stopitemize
  \stopitemize

\SlideTitle{Toolchain building utilities (3)}
Many root filesystem building systems also allow the construction of
a cross-compiling toolchain
\startitemize
\item {\bf Buildroot}
  \startitemize
  \item Makefile-based, has a Crosstool-NG back-end, maintained by the
    community
  \item \hyphenatedurl{http://www.buildroot.net}
  \stopitemize
\item {\bf PTXdist}
  \startitemize
  \item Makefile-based, uClibc or glibc, maintained mainly by {\em Pengutronix}
  \item \hyphenatedurl{http://www.pengutronix.de/software/ptxdist/index_en.html}
  \stopitemize
\item {\bf OpenEmbedded}
  \startitemize
  \item The feature-full, but more complicated building system
  \item \hyphenatedurl{http://www.openembedded.org/}
  \stopitemize
\stopitemize

  \SlideTitle{Crosstool-NG: installation and usage}
  \startitemize
  \item Installation of Crosstool-NG can be done system-wide, or just locally in
    the source directory. For local installation:
\starttyping
./configure --enable-local
make
make install
\stoptyping
  \item Some sample configurations for various architectures are
    available in
    samples, they can be listed using
\starttyping
./ct-ng list-samples
\stoptyping
  \item To load a sample configuration
\starttyping
./ct-ng <sample-name>
\stoptyping
  \item To adjust the configuration
\starttyping
./ct-ng menuconfig
\stoptyping
  \item To build the toolchain
\starttyping
./ct-ng build
\stoptyping
  \stopitemize

  \SlideTitle{Toolchain contents}
  \startitemize
  \item The cross compilation tool binaries, in \type{bin/}
    \startitemize
    \item This directory can be added to your \type{PATH} to ease
      usage of the toolchain
    \stopitemize
  \item One or several {\em sysroot}, each containing
    \startitemize
    \item The C library and related libraries, compiled for the target
    \item The C library headers and kernel headers
    \stopitemize
  \item There is one {\em sysroot} for each variant: toolchains can be
    {\em multilib} if they have several copies of the C library for
    different configurations (for example: ARMv4T, ARMv5T, etc.)
    \startitemize
    \item CodeSourcery ARM toolchain are multilib, the sysroots are in
      \type{arm-none-linux-gnueabi/libc/},
      \type{arm-none-linux-gnueabi/libc/armv4t/},
      \type{arm-none-linux-gnueabi/libc/thumb2}
    \item Crosstool-NG toolchains are never multilib, the sysroot is
      in \type{arm-unknown-linux-uclibcgnueabi/sysroot}
    \stopitemize
  \stopitemize
