  \SlideTitle{Definition (1)}
  \startitemize
  \item The usual development tools available on a GNU/Linux
    workstation is a {\bf native toolchain}
  \item This toolchain runs on your workstation and generates code for
    your workstation, usually x86
  \item For embedded system development, it is usually impossible or not
    interesting to use a native toolchain
    \startitemize
    \item The target is too restricted in terms of storage and/or memory
    \item The target is very slow compared to your workstation
    \item You may not want to install all development tools on your target.
    \stopitemize
  \item Therefore, {\bf cross-compiling toolchains} are generally
    used. They run on your workstation but generate code for your
    target.
  \stopitemize

  \IncludePicture[horizontal][slides/sysdev-toolchains-definition/cross-toolchain.pdf]{Definition (2)}

  \SlideTitle{Machines in build procedures}
  \startitemize
  \item Three machines must be distinguished when discussing toolchain creation
    \startitemize
    \item The {\bf build} machine, where the toolchain is built.
    \item The {\bf host} machine, where the toolchain will be executed.
    \item The {\bf target} machine, where the binaries created by the
      toolchain are executed.
    \stopitemize
  \item Four common build types are possible for toolchains
  \stopitemize

  \IncludePicture[horizontal][slides/sysdev-toolchains-definition/toolchain-build-types.pdf]{Different toolchain build procedures}

  \IncludePicture[horizontal][slides/sysdev-toolchains-definition/components.pdf]{Components}

  \SlideTitle{Binutils}
  \startitemize
  \item {\bf Binutils} is a set of tools to generate and manipulate
    binaries for a given CPU architecture
    \startitemize
    \item \type{as}, the assembler, that generates binary code from
      assembler source code
    \item \type{ld}, the linker
    \item \type{ar}, \type{ranlib}, to generate \type{.a} archives,
      used for libraries
    \item \type{objdump}, \type{readelf}, \type{size}, \type{nm},
      \type{strings}, to inspect binaries. Very useful analysis tools!
    \item \type{strip}, to strip useless parts of binaries in order to
      reduce their size
    \stopitemize
  \item \hyphenatedurl{http://www.gnu.org/software/binutils/}
  \item GPL license
  \stopitemize

  \SlideTitle{Kernel headers (1)}
    \placefigure[right,none]{}{
    \externalfigure[slides/sysdev-toolchains-definition/kernel-headers.pdf][width=0.2\textwidth]
    }
    \startitemize
    \item The C library and compiled programs needs to interact with the kernel
      \startitemize
      \item Available system calls and their numbers
      \item Constant definitions
      \item Data structures, etc.
      \stopitemize
    \item Therefore, compiling the C library requires kernel headers, and many
      applications also require them.
    \item Available in \type{<linux/...>} and \type{<asm/...>} and a few
      other directories corresponding to the ones visible in
      \type{include/} in the kernel sources
    \stopitemize

  \SlideTitle{Kernel headers (2)}
  \startitemize
  \item System call numbers, in \type{<asm/unistd.h>}
\starttyping
#define __NR_exit         1
#define __NR_fork         2
#define __NR_read         3
\stoptyping
  \item Constant definitions, here in \type{<asm-generic/fcntl.h>},
    included from \type{<asm/fcntl.h>}, included from
    \type{<linux/fcntl.h>}
\starttyping
#define O_RDWR 00000002
\stoptyping
\item Data structures, here in \type{<asm/stat.h>}
\starttyping
struct stat {
    unsigned long st_dev;
    unsigned long st_ino;
    [...]
};
\stoptyping
\stopitemize

  \SlideTitle{Kernel headers (3)}
  \startitemize
  \item The kernel-to-userspace ABI is {\bf backward compatible}
    \startitemize
    \item Binaries generated with a toolchain using kernel headers
      older than the running kernel will work without problem, but
      won't be able to use the new system calls, data structures, etc.
    \item Binaries generated with a toolchain using kernel headers
      newer than the running kernel might work on if they don't use
      the recent features, otherwise they will break
    \item Using the latest kernel headers is not necessary, unless
      access to the new kernel features is needed
    \stopitemize
  \item The kernel headers are extracted from the kernel sources using
    the \type{headers_install} kernel Makefile target.
  \stopitemize

  \SlideTitle{GCC}
  \placefigure[right,none]{}{
  \externalfigure[slides/sysdev-toolchains-definition/gcc.png]
  }
    \startitemize
    \item GNU Compiler Collection, the famous free software compiler
    \item Can compile C, C++, Ada, Fortran, Java, Objective-C,
      Objective-C++, and generate code for a large number of CPU
      architectures, including ARM, AVR, Blackfin, CRIS, FRV, M32, MIPS,
      MN10300, PowerPC, SH, v850, i386, x86\_64, IA64, Xtensa, etc.
    \item \hyphenatedurl{http://gcc.gnu.org/}
    \item Available under the GPL license, libraries under the LGPL.
    \stopitemize

  \SlideTitle{C library}
    \placefigure[right,none]{}{
        \externalfigure[slides/sysdev-toolchains-definition/c-library.pdf][width=0.2\textwidth]
    }
    \startitemize
    \item The C library is an essential component of a Linux system
      \startitemize
      \item Interface between the applications and the kernel
      \item Provides the well-known standard C API to ease application
        development
      \stopitemize
    \item Several C libraries are available:\\
      {\em glibc}, {\em uClibc}, {\em eglibc}, {\em dietlibc}, {\em
        newlib}, etc.
    \item The choice of the C library must be made at the time of the
      cross-compiling toolchain generation, as the GCC compiler is
      compiled against a specific C library.
    \stopitemize
