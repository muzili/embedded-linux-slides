
  \SlideTitle{Cross-compiling the kernel}
  When you compile a Linux kernel for another CPU architecture
  \startitemize
  \item Much faster than compiling natively, when the target system is
    much slower than your GNU/Linux workstation.
  \item Much easier as development tools for your GNU/Linux
    workstation are much easier to find.
  \item To make the difference with a native compiler, cross-compiler
    executables are prefixed by the name of the target system,
    architecture and sometimes
    library. Examples:\\
    \type{mips-linux-gcc}, the prefix is \type{mips-linux-}\\
    \type{arm-linux-gnueabi-gcc}, the prefix is \type{arm-linux-gnueabi-}
  \stopitemize

  \SlideTitle{Specifying cross-compilation (1)}

  The CPU architecture and cross-compiler prefix are defined through
  the \type{ARCH} and \type{CROSS_COMPILE} variables in the toplevel
  Makefile.

  \startitemize
  \item \type{ARCH} is the name of the architecture. It is defined by
    the name of the subdirectory in \type{arch/} in the kernel sources
    \startitemize
    \item Example: \type{arm} if you want to compile a kernel for
          the \type{arm} architecture.
    \stopitemize
  \item \type{CROSS_COMPILE} is the prefix of the cross compilation
    tools
    \startitemize
    \item Example: \type{arm-linux-} if your compiler is \type{arm-linux-gcc}
    \stopitemize
  \stopitemize

  \SlideTitle{Specifying cross-compilation (2)}

  Two solutions to define \type{ARCH} and \type{CROSS_COMPILE}:

  \startitemize
  \item Pass \type{ARCH} and \type{CROSS_COMPILE} on the \type{make}
    command line: \\
    \type{make ARCH=arm CROSS_COMPILE=arm-linux- ...} \\
    Drawback: it is easy to forget to pass these variables when
    you run any \type{make} command, causing your build and
    configuration to be screwed up.
  \item Define \type{ARCH} and \type{CROSS_COMPILE} as environment
    variables: \\
    \type{export ARCH=arm} \\
    \type{export CROSS_COMPILE=arm-linux-} \\
    Drawback: it only works inside the current
    shell or terminal. You could put these settings in a file
    that you source every time you start working on the project.
    If you only work on a single architecture with always the 
    same toolchain, you could even put these settings in your
    \type{~/.bashrc} file to make them permanent and visible from
    any terminal.
  \stopitemize

  \SlideTitle{Predefined configuration files}
  \startitemize
  \item Default configuration files available, per board or per-CPU
    family
    \startitemize
    \item They are stored in \type{arch/<arch>/configs/}, and are
      just minimal \type{.config} files
    \item This is the most common way of configuring a kernel for
      embedded platforms
    \stopitemize
  \item Run \type{make help} to find if one is available for your
    platform
  \item To load a default configuration file, just run\\
    \type{make acme_defconfig}
    \startitemize
    \item This will overwrite your existing \type{.config} file!
    \stopitemize
  \item To create your own default configuration file
    \startitemize
    \item \type{make savedefconfig}, to create a minimal
      configuration file
    \item \type{mv defconfig arch/<arch>/configs/myown_defconfig}
    \stopitemize
  \stopitemize

  \SlideTitle{Configuring the kernel}
  \startitemize
  \item After loading a default configuration file, you can adjust the
    configuration to your needs with the normal \type{xconfig},
    \type{gconfig} or \type{menuconfig} interfaces
  \item You can also start the configuration from scratch without
    loading a default configuration file
  \item As the architecture is different from your host architecture
    \startitemize
    \item Some options will be different from the native configuration
      (processor and architecture specific options, specific drivers,
      etc.)
    \item Many options will be identical (filesystems, network
      protocol, architecture-independent drivers, etc.)
    \stopitemize
  \item Make sure you have the support for the right CPU, the right
    board and the right device drivers.
  \stopitemize

  \SlideTitle{Building and installing the kernel}
  \startitemize
  \item Run \type{make}
  \item Copy the final kernel image to the target storage
    \startitemize
    \item can be \type{uImage}, \type{zImage}, \type{vmlinux},
      \type{bzImage} in \type{arch/<arch>/boot}
    \stopitemize
  \item \type{make install} is rarely used in embedded development, as the
    kernel image is a single file, easy to handle
    \startitemize
    \item It is however possible to customize the make install
      behaviour in \type{arch/<arch>/boot/install.sh}
    \stopitemize
  \item \type{make modules_install} is used even in embedded
    development, as it installs many modules and description files
    \startitemize
    \item \type{make INSTALL_MOD_PATH=<dir>/ modules_install}
    \item The \type{INSTALL_MOD_PATH} variable is needed to install
      the modules in the target root filesystem instead of your host
      root filesystem.
    \stopitemize
  \stopitemize

  \SlideTitle{Kernel command line}
  \startitemize
  \item In addition to the compile time configuration, the kernel
    behaviour can be adjusted with no recompilation using the {\bf
      kernel command line}
  \item The kernel command line is a string that defines various
    arguments to the kernel
    \startitemize
    \item It is very important for system configuration
    \item \type{root=} for the root filesystem (covered later)
    \item \type{console=} for the destination of kernel messages
    \stopitemize
  \item This kernel command line is either
    \startitemize
    \item Passed by the bootloader. In U-Boot, the contents of the
      \type{bootargs} environment variable is automatically passed to the
      kernel
    \item Built into the kernel, using the \type{CONFIG_CMDLINE} option.
    \stopitemize
  \stopitemize
