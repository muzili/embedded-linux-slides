  \SlideTitle{ABI}
  \startitemize
  \item When building a toolchain, the ABI used to generate binaries
    needs to be defined
  \item ABI, for {\em Application Binary Interface}, defines the
    calling conventions (how function arguments are passed, how the
    return value is passed, how system calls are made) and the
    organization of structures (alignment, etc.)
  \item All binaries in a system must be compiled with the same ABI,
    and the kernel must understand this ABI.
  \item On ARM, two main ABIs: {\em OABI} and {\em EABI}
    \startitemize
    \item Nowadays everybody uses {\em EABI}
    \stopitemize
  \item On MIPS, several ABIs: {\em o32, o64, n32, n64}
  \item \hyphenatedurl{http://en.wikipedia.org/wiki/Application_Binary_Interface}
  \stopitemize

  \SlideTitle{Floating point support}
  \startitemize
  \item Some processors have a floating point unit, some others do not.
    \startitemize
    \item For example, many ARMv4 and ARMv5 CPUs do not have a
      floating point unit.  Since ARMv7, a VFP unit is mandatory.
    \stopitemize
  \item For processors having a floating point unit, the toolchain
    should generate {\em hard float} code, in order to use the
    floating point instructions directly
  \item For processors without a floating point unit, two solutions
    \startitemize
    \item Generate {\em hard float code} and rely on the kernel to
      emulate the floating point instructions. This is very slow.
    \item Generate {\em soft float code}, so that instead of
      generating floating point instructions, calls to a userspace
      library are generated
    \stopitemize
  \item Decision taken at toolchain configuration time
  \item Also possible to configure which floating point unit should be used
  \stopitemize

  \SlideTitle{CPU optimization flags}
  \startitemize
  \item A set of cross-compiling tools is specific to a CPU architecture (ARM,
    x86, MIPS, PowerPC)
  \item However, with the \type{-march=}, \type{-mcpu=},
    \type{-mtune=} options, one can select more precisely the
    target CPU type
    \startitemize
    \item For example, \type{-march=armv7 -mcpu=cortex-a8}
    \stopitemize
  \item At the toolchain compilation time, values can be chosen. They are used:
    \startitemize
    \item As the default values for the cross-compiling tools, when no
      other \type{-march}, \type{-mcpu}, \type{-mtune} options are
      passed
    \item To compile the C library
    \stopitemize
  \item Even if the C library has been compiled for armv5t, it doesn't
    prevent from compiling other programs for armv7
  \stopitemize
