\SlideTitle{What is embedded Linux?}
\startitemize
\item Embedded Linux is the usage of the {\bf Linux kernel} and various
  {\bf open-source} components in embedded systems
\item Android is not embedded Linux
\stopitemize

\SlideTitle{Re-using components}
\startitemize
\item The key advantage of Linux and open-source in embedded systems
  is the {\bf ability} to re-use components
\item The open-source ecosystem already provides many components for
  standard features, from hardware support to network protocols,
  going through multimedia, graphic, cryptographic libraries, etc.
\item As soon as a hardware device, or a protocol, or a feature is
  wide-spread enough, high chance of having open-source components
  that support it.
\item Allows to quickly design and develop complicated products,
  based on existing components.
\item No-one should re-develop yet another operating system kernel,
  TCP/IP stack, USB stack or another graphical toolkit library.
\item {\bf Allows to focus on the added value of your product.}
\stopitemize

\SlideTitle{Low cost}
\startitemize
\item Free software can be duplicated on as many devices as you
  want, free of charge.
\item If your embedded system uses only free software, you can
  reduce the cost of software licenses to zero. Even the development
  tools are free, unless you choose a commercial embedded Linux
  edition.
\item {\bf Allows to have a higher budget for the hardware or to
  increase the company’s skills and knowledge}
\stopitemize

\SlideTitle{Full control}
\startitemize
\item With open-source, you have the source code for all components
  in your system
\item Allows unlimited modifications, changes, tuning, debugging,
  optimization, for an unlimited period of time
\item Without locking or dependency from a third-party vendor
  \startitemize
  \item To be true, non open-source components must be avoided when
    the system is designed and developed
  \stopitemize
\item {\bf Allows to have full control over the software part of
  your system}
\stopitemize

\SlideTitle{Quality}
\startitemize
\item Many open-source components are widely used, on millions of
  systems
\item Usually higher quality than what an in-house development can
  produce, or even proprietary vendors
\item Of course, not all open-source components are of good quality,
  but most of the widely-used ones are.
\item {\bf Allows to design your system with high-quality components
  at the foundations}
\stopitemize

\SlideTitle{Eases testing of new features}
\startitemize
\item Open-source being freely available, it is easy to get a piece
  of software and evaluate it
\item Allows to easily study several options while making a choice
\item Much easier than purchasing and demonstration procedures
  needed with most proprietary products
\item {\bf Allows to easily explore new possibilities and solutions}
\stopitemize

\SlideTitle{Community support}
\startitemize
\item Open-source software components are developed by communities
  of developers and users
\item This community can provide a high-quality support: you can
  directly contact the main developers of the component you are
  using. The likelyhood of getting an answer doesn't depend what
  company you work for.
\item Often better than traditional support, but one needs to
  understand how the community works to properly use the community
  support possibilities
\item {\bf Allows to speed up the resolution of problems when
  developing your system}
\stopitemize

\SlideTitle{Processor and architecture (1)}
\startitemize
\item The Linux kernel and most other architecture-dependent component support a
  wide range of 32 and 64 bits architectures
  \startitemize
  \item x86 and x86-64, as found on PC platforms, but also embedded systems
    (multimedia, industrial)
  \item ARM, with hundreds of different SoC (multimedia, industrial)
  \item PowerPC (mainly real-time, industrial applications)
  \item MIPS (mainly networking applications and multimedia)
  \item SuperH (mainly set top box and multimedia applications)
  \item Blackfin (DSP architecture)
  \item Microblaze (soft-core for Xilinx FPGA)
  \item Coldfire, SCore, Tile, Xtensa, Cris, FRV, AVR32, M32R
  \stopitemize
\stopitemize


\SlideTitle{Processor and architecture (2)}
\startitemize
\item Both MMU and no-MMU architectures are supported, even though
  no-MMU architectures have a few limitations.
\item Linux is not designed for small microcontrollers.
\item Besides the toolchain, the bootloader and the kernel, all
  other components are generally {\bf architecture-independent}
\stopitemize



\SlideTitle{RAM and storage}
\startitemize
\item {\bf RAM}: a very basic Linux system can work within 8 MB of
  RAM, but a more realistic system will usually require at least 32
  MB of RAM. Depends on the type and size of applications.
\item {\bf Storage}: a very basic Linux system can work within 4 MB
  of storage, but usually more is needed.
  \startitemize
  \item Flash storage is supported, both NAND and NOR flash, with
    specific filesystems
  \item Block storage including SD/MMC cards and eMMC is supported
  \stopitemize
\item Not necessarily interesting to be too restrictive on the
  amount of RAM/storage: having flexibility at this level allows to
  re-use as many existing components as possible.
\stopitemize



\SlideTitle{Communication}
The Linux kernel has support for many common communication
  busses
  \startitemize
  \item I2C/SPI/CAN/1-wire/SDIO/USB
  \item Ethernet, Wifi, Bluetooth, CAN, etc.
  \item IPv4, IPv6, TCP, UDP, SCTP, DCCP, etc.
  \item Firewalling, advanced routing, multicast
  \stopitemize

\SlideTitle{Types of hardware platforms}
\startitemize
\item {\bf Evaluation platforms} from the SoC vendor. Usually
  expensive, but many peripherals are built-in. Generally unsuitable
  for real products.
\item {\bf Component on Module}, a small board with only
  CPU/RAM/flash and a few other core components, with connectors to
  access all other peripherals. Can be used to build end products
  for small to medium quantities.
\item {\bf Community development platforms}, a new trend to make a
  particular SoC popular and easily available. Those are
  ready-to-use and low cost, but usually have less peripherals than
  evaluation platforms. To some extent, can also be used for real
  products.
\item {\bf Custom platform}. Schematics for evaluation boards or
  development platforms are more and more commonly freely available,
  making it easier to develop custom platforms.
\stopitemize



\SlideTitle{Criteria for choosing the hardware}
\startitemize
\item Make sure the hardware you plan to use is already supported by
  the Linux kernel, and has an open-source bootloader, especially
  the SoC you’re targeting.
\item Having support in the official versions of the projects
  (kernel, bootloader) is a lot better: quality is better, and new
  versions are available.
\item Some SoC vendors and/or board vendors do not contribute their
  changes back to the mainline Linux kernel. Ask them to do so, or
  use another product if you can. A good measurement is to see the
  delta between their kernel and the official one.
\item {\bf Between properly supported hardware in the official Linux
  kernel and poorly-supported hardware, there will be huge
  differences in development time and cost.}
\stopitemize


% To include a full slide picute.
\IncludePicture
  [horizontal]
  [slides/sysdev-intro/global-architecture.pdf] % Name of the image
  {Global architecture} % Title of the slide


  \SlideTitle{Software components}
  \startitemize
  \item {\bf Cross-compilation toolchain}
    Compiler that runs on the development machine, but generates
    code for the target
  \item {\bf Bootloader}
    Started by the hardware, responsible for basic
      initialization, loading and executing the kernel
  \item {\bf Linux Kernel}
    Contains the process and memory management, network stack,
      device drivers and provides services to userspace applications
  \item {\bf C library}
    The interface between the kernel and the userspace
      applications
  \item {\bf Libraries and applications}
     Third-party or in-house
  \stopitemize

  \SlideTitle{Embedded Linux work}

  Several distinct tasks are needed when deploying embedded Linux in a
  product:

  \startitemize
  \item {\bf Board Support Package development}
    \startitemize
    \item A BSP contains a bootloader and kernel with the suitable
      device drivers for the targeted hardware
    \item Purpose of our {\em Kernel Development} training
    \stopitemize
  \item {\bf System integration}
    \startitemize
    \item Integrate all the components, bootloader, kernel,
      third-party libraries and applications and in-house applications
      into a working system
    \item Purpose of {\em this} training
    \stopitemize
  \item {\bf Development of applications}
    \startitemize
    \item Normal Linux applications, but using specifically chosen
      libraries
    \stopitemize
  \stopitemize

