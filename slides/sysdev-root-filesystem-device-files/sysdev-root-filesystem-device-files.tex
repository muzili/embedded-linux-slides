  \SlideTitle{Devices}
  \startitemize
  \item One of the kernel important role is to {\bf allow applications
      to access hardware devices}
  \item In the Linux kernel, most devices are presented to userspace
    applications through two different abstractions
    \startitemize
    \item {\bf Character} device
    \item {\bf Block} device
    \stopitemize
  \item Internally, the kernel identifies each device by a triplet of
    information
    \startitemize
    \item {\bf Type} (character or block)
    \item {\bf Major} (typically the category of device)
    \item {\bf Minor} (typically the identifier of the device)
    \stopitemize
  \stopitemize

  \SlideTitle{Types of devices}
  \startitemize
  \item Block devices
    \startitemize
    \item A device composed of fixed-sized blocks, that can be read
      and written to store data
    \item Used for hard disks, USB keys, SD cards, etc.
    \stopitemize
  \item Character devices
    \startitemize
    \item Originally, an infinite stream of bytes, with no beginning,
      no end, no size. The pure example: a serial port.
    \item Used for serial ports, terminals, but also sound cards,
      video acquisition devices, frame buffers
    \item Most of the devices that are not block devices are
      represented as character devices by the Linux kernel
    \stopitemize
  \stopitemize

  \SlideTitle{Devices: everything is a file}
  \startitemize
  \item A very important Unix design decision was to represent most of
    the ``system objects'' as files
  \item It allows applications to manipulate all “system objects” with
    the normal file API (\type{open}, \type{read}, \type{write},
    \type{close}, etc.)
  \item So, devices had to be represented as files to the applications
  \item This is done through a special artifact called a {\bf device
      file}
  \item It is a special type of file, that associates a file name
    visible to userspace applications to the triplet {\em (type,
      major, minor)} that the kernel understands
  \item All {\em device files} are by convention stored in the
    \type{/dev} directory
  \stopitemize

\SlideTitle{Device files examples}

Example of device files in a Linux system

\small
\starttyping
$ ls -l /dev/ttyS0 /dev/tty1 /dev/sda1 /dev/sda2 /dev/zero
brw-rw---- 1 root disk    8,  1 2011-05-27 08:56 /dev/sda1
brw-rw---- 1 root disk    8,  2 2011-05-27 08:56 /dev/sda2
crw------- 1 root root    4,  1 2011-05-27 08:57 /dev/tty1
crw-rw---- 1 root dialout 4, 64 2011-05-27 08:56 /dev/ttyS0
crw-rw-rw- 1 root root    1,  5 2011-05-27 08:56 /dev/zero
\stoptyping

Example C code that uses the usual file API to write data to a serial port

\starttyping
int fd;
fd = open("/dev/ttyS0", O_RDWR);
write(fd, "Hello", 5);
close(fd);
\stoptyping

  \SlideTitle{Creating device files}
  \startitemize
    \item On a basic Linux system, the device files have to be created
    manually using the \type{mknod} command
    \startitemize
    \item \type{mknod /dev/<device> [c|b] major minor}
    \item Needs root privileges
    \item Coherency between device files and devices handled by the
      kernel is left to the system developer
    \stopitemize
  \item On more elaborate Linux systems, mechanisms can be added to
    create/remove them automatically when devices appear and disappear
    \startitemize
    \item \type{devtmpfs} virtual filesystem, since kernel 2.6.32
    \item \type{udev} daemon, solution used by desktop and server Linux
      systems
    \item \type{mdev} program, a lighter solution than \type{udev}
    \stopitemize
  \stopitemize
